% Author: Valentin Churavy vchuravy@mit.edu
\documentclass{beamer}
\usepackage[T1]{fontenc}
\usepackage[utf8]{inputenc}

\RequirePackage[l2tabu, orthodox]{nag}

\usepackage{graphicx}
\usepackage{epstopdf}
\usepackage{tabularx}
\usepackage{epigraph}

\usepackage{lmodern}
\usepackage{enumerate}

\usepackage{url}

\usepackage{microtype}

%% language (hyphenation, date format, ...)
\usepackage[british]{babel}
\usepackage[babel]{csquotes}

%% math packages
\usepackage{%
  amsmath,      % AMS math package
  amsfonts,     % AMS math fonts
  amssymb       % AMS symbols
}

\usetheme[leftcorneredgy]{VCMIT}

%%Bibliography
\usepackage[backend=biber, style=numeric]{biblatex}
\bibliography{references.bib}

%% Document Details
\title{Concolic Execution for Fuzzing in Julia}
\subtitle{Final project for 6.858 (Spring 2018)}
\author{Valentin Churavy}
\institute[CSAIL]{JuliaLab@CSAIL}
\date{16. May 2018}

\begin{document}
  \begin{frame}[plain]
    \titlepage
  \end{frame}
  \begin{frame}{Table of Contents}
    \tableofcontents[hideallsubsections]
  \end{frame}
  \section{Concolic Execution}
  \begin{frame}{Concolic Execution}
      \begin{block}{Welcome to Lab 3}
          \pause
          ... but in Julia!
      \end{block}
\pause
      \begin{block}{General strategy}
          \begin{itemize}[<+->]
          \item Generate symbolic trace of the program for a given input
          \item Feed symbolic trace to Z3 to generate inputs that cause exploration of unvisited branches
          \item Iterate until all branches are visited, record which branches throw exceptions
      \end{itemize}
          Based on \cite{Godefroid2005-ld, Cadar2006-hs}
      \end{block}
  \end{frame}
  \section{Julia in brief}
  \begin{frame}{Julia in brief}
      \begin{block}{}
          Julia \cite{Bezanson2012-iw} is a dynamic high-performance language, it looks like Python, runs like Fortran,
      and talks like Lisp. It has interesting language design properties, like a rich type
          system, multiple dispatch, macros, and staged programming. It uses a compiler based on LLVM\cite{Lattner2004-dv}.
      \end{block}

      \pause
      \begin{exampleblock}{This is not a Julia tutorial!}
          I am going to talk about a couple of advanced concepts and use a
          development version of Jula with experimental tooling!

          If you want to learn more about Julia, ask me later!
      \end{exampleblock}

      \pause
      \begin{block}{AD in 5s}
          Example time!
      \end{block}
  \end{frame}
  \section{Implementation}
  \begin{frame}{Implementation}
      \begin{itemize}[<+->]
        \item Goal: Be able to deal with arbitrary user code
        \item Trace an invocation of a function, e.g. record recursivly,
              which calls occur and which input + output arguments they have.
        \item Mark/Taint input arguments as symbolic
        \item Issue: The trace is linearized and no longer contains control-flow
        \item Solution: Rewrite and rejit user-provided functions and instert calls
              to the assert function, which records conditionals that are used as part
              of the control-flow.
        \item Filter concolic trace to get a trace that only depends on the tainted arguments
        \item Cut trace at branch and negate branch condition
        \item Translate symbolic trace to SMT2/Z3, if satisfiable use the input value to continue exploration
    \end{itemize}
  \end{frame}
  \begin{frame}{Solving the type-constraint problem with Cassette}
      One of the challenges was to be able to taint arbitrary user-code, including user code that has type-constraints. \pause
      \begin{exampleblock}{Cassette}
          \begin{itemize}[<+->]
            \item Cassette.jl \cite{Revels_undated-st} developed here at MIT provides a framework to extend Julia with non-standard execution models, like AD or symbolic execution, through contextual dispatch
            \item User-code is executed within a \emph{context} in which the context author can provide alternative \emph{primitives} and can dynamically inject passes into the compiler
            \item Usage for this project
                \begin{itemize}[<+->]
                \item Trace generation through pre-/post-hooks on every function call
                \item Rewrite IR to inject assert statements
                \item Taint propagation through the metadata system of Cassette
                \item Context specific primitives of operations (like addition) to propagate taint or to create tain (rand)
            \end{itemize}
        \end{itemize}
      \end{exampleblock}
  \end{frame}
  \begin{frame}
      \begin{block}{Extensible design: rand}
          One problem in concolic execution is how to deal with random variables, concolic execution depends on being able to \emph{deterministically} execute programs.
          \emph{rand} is implemented as a primitive that generates taint and the return value of a seen execution of \emph{rand} can be controlled and during fuzzing
          Z3 will generate values for \emph{rand} much in the same way it does for other inputs.

          This could be extended to other foreign-calls like reading from the filesystem.
      \end{block}
  \end{frame}
  \section{Examples}
  \begin{frame}{Examples}
    \begin{itemize}
        \item Proof some properties
        \item Fuzzing!
        \item Loops, structs, random variables
     \end{itemize}
  \end{frame}
  \section{Limitations}
  \begin{frame}{Future-work: Features}
      \begin{itemize}[<+->]
          \item Reasoning about the type-domain
          \item Taint analysis through Arrays
          \item Taint analysis with non-integer inputs
            \begin{itemize}
                \item Strings
                \item Structs
                \item Floating-point values
            \end{itemize}
        \item Arithmetic on Bool
      \end{itemize}
  \end{frame}
  \begin{frame}{Future-work: Tooling}
      \begin{itemize}[<+->]
          \item Using the Klee query language
          \item Maybe deeper integration with Klee\cite{Cadar_Dunbar_Engler}
          \item Only intercept intrinsics and built-ins instead of higher
                level methods
          \item Better analysis of function to taint usage of global
                variables and foreigncalls
      \end{itemize}
  \end{frame}
  \begin{frame}
      \begin{block}{Where to find this work}
          \url{https://github.com/vchuravy/ConcolicFuzzer.jl}
      \end{block}
      \pause
      \begin{block}{If you want to learn more about Julia!}
          \begin{enumerate}
              \item Write me an e-mail or go to \url{https://julialang.org}
              \item Find me in 32-G785
            \end{enumerate}
      \end{block}
  \end{frame}
  \begin{frame}[allowframebreaks]{Bibliography}
      \printbibliography
\end{frame}
\end{document}
