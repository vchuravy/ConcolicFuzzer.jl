\documentclass{article} % For LaTeX2e

\usepackage[utf8]{inputenc}
\usepackage[english]{babel}
\usepackage{nips13submit_e,times}
\usepackage{url}
\usepackage{graphicx}
% Code listings
\usepackage{minted}
\usepackage[backend=biber, style=numeric]{biblatex}

\usepackage{varioref}
\usepackage[unicode, naturalnames]{hyperref} % For pdf-documents
\usepackage{cleveref}

\usepackage{microtype} % Include and forget
\usepackage{scrhack} % Because of minted...
\title{
Concolic Execution in Julia\\
\small{Final project report for 6.858 (2018)}
}

\author{
Valentin Churavy\\
JuliaLab@CSAIL\\
\texttt{vchuravy@csail.mit.edu}
}

\newcommand{\fix}{\marginpar{FIX}}
\newcommand{\new}{\marginpar{NEW}}

\nipsfinalcopy % Uncomment for camera-ready version
\bibliography{references.bib}

\begin{document}

\maketitle

\section{A short introduction to Julia}
The Julia Language \cite{Bezanson2012-iw} is a dynamic programming language that aims to
square the circle of provide high-performance whilst also being high-level. The language
is build around a JIT compiler based on LLVM \cite{Lattner2004-dv}. While the language 
semantics are dynamic, extensive type-inference let's Julia compile and optimize functions 
that are proven to be type instable. Julia's unit of compilation are method and each method
is specialised on a given set of argument types. While most of the compiler is written in C/C++,
the language implementation and runtime is written in Julia itself. And bootstrapped on a very
small set of intrinsics and built-in methods. Julia has a rich type system, metaprogramming support,
uses multiple-dispatch, and has a large set of foreign-function interfaces.

\section{Goals}
I set out to implement a concolic execution engine for Julia, with a focus on proving application
specific invariances, I followed the general ideas set out in the third problem set and papers
\cite{Godefroid2005-ld, Cadar2006-hs}. My original goals as layed out in the proposal were:

\begin{enumerate}
    \item Given a particular method to analyze, recursively build a graph of methods
          that are called (terminating the recursion at intrinsic functions)
    \item Collect set of asserts (asserts can either be assumptions or invariants to be proven)
    \item Implement symbolic execution scheme for primitive data-types and intrinsic functions
    \item Extend symbolic execution scheme for user-defined immutable and mutable datatypes
    \item Propagate set of assumptions through the call graph, instead of attempting to prove general correctness
    \item Add concolic execution for all cases that we can’t handle symbolically
\end{enumerate}

\section{Implementation}
Julia has several different intermediate representation on which one could build a symbolic execution system.
The important ones are the untyped AST, the typed AST and LLVM IR. The latter would provide an interesting
opportunity to integrate with KLEE \cite{Cadar_Dunbar_Engler}. The untyped/typed AST is currently under heavy
development and quite a complex target. After initial experimentation I decided against building a symbolic
execution system for Julia and narrowed the focus of this project. The approach I finally decided to go with
is a taint-based concolic first approach, that suits itself well for guided fuzzying.

For both tainting and trace generation I use an extension to the Julia language that provides contextual dispatch
called Cassette \cite{Revels_undated-st}. Cassette allows for non-standard execution of Julia code and the dynamic
insertion of code-transformation passes to compile arbitrary user code. The motivation for Cassette originally
originated in the context of automatic-differentiation \cite{noauthor_undated-iv}. As an example take the
following code-snippet:

\begin{minted}{julia}
    # D is a pair of the concrete value and the derivative
    struct D <: Number
        f::Tuple{Float64, Float64}
    end
    import Base: +, /, convert, promote_rule
    +(x::D, y::D) = D(x.f .+ y.f)
    /(x::D, y::D) = D((x.f[1]/y.f[1], (y.f[1]*x.f[2] - x.f[1]*y.f[2])/y.f[1]^2))
    convert(::Type{D}, x::Real) = D((x,zero(x)))
    promote_rule(::Type{D}, ::Type{<:Number}) = D

    function Babylonian(x; N = 10)
        t = (1+x)/2
        for i = 2:N
            t=(t + x/t)/2
        end
        t
    end
    x=pi; Babylonian(D((x,1))), (sqrt(x), 0.5/sqrt(x))
    # (D((1.7724538509055159, 0.28209479177387814)),
    #    (1.7724538509055159, 0.28209479177387814))
\end{minted}
Using this minimal implementation of forward-mode auto-differentation we can calculate the derivative of
any generic user-defined function (as long as they only use plus and division). In this examle we use the 
babylonian method for the calculation of the square-root. This approach works spendidly, but trouble arises if they user provides a function
that is type-constrained, e.g. \mint{julia}|Babylonian(x::Float64, N = 10)|, then we no longer thread our Dual
number into the user-provided function. A similar problem would arise in the context of symbolic-execution,
while we easily could define a symbolic type that records all operations done on it, every type annotation
would create a blackbox. Contextual-dispatch allows us to sidestep this issue, by defining a non-standard
execution in which we rewrite type-annotated functions to be permissive for our Dual type. In actuality
Cassette defines a metadata propagation system that is used in situation like this.
Another important feature is that contextual-dispatch occurs within a context and does not modify the code
running outside the context, code transformations are not visible to the user.

\subsection{Concolic execution -- Generation of traces and tainting}
Initally we mark input/taint arguments -- in Cassette parlance, they are boxes with associated metadata -- and
propagate them through the function. Each operation that takes a tainted argument has to mark it's output as
tainted (see \cref{sec:Limitations} for the limitations of the current prototype). In parallel we record a
trace of the current execution in form of nested callsites (callsites are function names, with arguments and outputs)
This execution occurs over a set of concrete values, but the trace are concolic, e.g. if a value was tainted only
the symbolic variable name is recorded, otherwise the concrete value is recorded.

The recorded trace can then be flattened (removal of all non-leaf nodes, this action is corresponding to inlining)
and filtered (removal off all nodes that have not at least one symbolic component), these two proceddurs yield
a linearized control-flow stream. Streams have the property that they still contain all operations that use tainted
values, e.g. values that are under an attacker control. It is possible to add additional sources of taint for an
example see \cref{subsec:rand}.

\subsection{Assertions}
Since streams and traces follow concrete executions of a function, they no longer contain information about the
control flow. ConcolicFuzzer.jl therefore provides the \emph{assert} function to manually insert user-defined
invariances. Since manually annotating every if-condition would be a nuisance, it uses a Cassette pass to
insert an assertion before every branch. Branches are represented as the \emph{goto\_if\_not} statement in the Julia AST.
This is implemented in the \emph{InsertAssertsPass}, and allows us to capture the conditional of the branch and
its value in our trace.

\subsection{Other sources of taint: rand}
\label{subsec:rand}
Another source of control-flow variation and a case in which the input based tracing is too limited, in contrast
to full symbolic execution, is the usage of \emph{rand} to generate random values within a traced function.

\emph{rand} is defined as a primitive function in the context of ConcolicFuzzer.jl. The primitive taints
the return value and checks within the trace local context if a value is prescribed.

The latter feature allows for deterministics execution of traces and it provides an avenue for the fuzzer to
drive execution, the values coming from \emph{rand} are represented as free variables SMT2 and the fuzzer uses
Z3 to generate
values that will cause control-flow divergence within the trace. This approach is extensible for other sources
of non-determentistic values (that are not controlled through arguments), like calls to the operating system,
 the runtime system, or arbitrary libraries that are called through a FFI.

\subsection{Using symbolic traces to generate input values}
The current implmentation only supports a limited number of integer programs, due to the fact
that the translation of traces/streams to SMT2 is rudimentary. In the linearized stream assertions
occur sequentially. The main fuzzer loop starts at a depth of 0 with an initial guess at the parameters.
The execution either fails (in which case we found a bug) or succeeds and the program obtains a trace,
with n branches. Starting at the first branch we cut the subsequent trace and negate the branch, then we
convert the stream to SMT2 and let Z3 provide us a set of input arguments that satisfy the condition.
If it is unsatisfiable we have shown that we can't reach that branch in the opposite direction and we
remove it from the candidate list. If it was satisfiable we found a new candidate that we add to workqueue.
We continue to iterate until the workqueue is empty and we have explored all branches in the program that
are controllable by the input arguments.

\section{Limitations}
\label{sec:Limitations}
ConcolicFuzzer.jl is currently limited to integer programs and the two main bottleneck is that all primitive/leaf
functions need to be enumerated and they need to be translated to repsective SMT2 expressions. Further work needs
to be gone to use SMT2 BitVectors and add support for floating-point value. The original goal was to only
declare intrinsics (about 100) and built-in functions (about 24) as primitives, but for expedience sake a limited
set of integer operations was defined as primitives.

Additionally functions that return tuples and where callsites use desugaring to extract the values are unsupported.

\section{Conclusion}
The code is available on Github at \url{https://github.com/vchuravy/ConcolicFuzzer.jl} under the MIT license.
Please check the examples in the test-suite at \url{https://github.com/vchuravy/ConcolicFuzzer.jl/tree/master/test}.

To run the code please check the README, it contains the versions that of Julia and Cassette that this code,
was developed against.

\printbibliography

\end{document}
